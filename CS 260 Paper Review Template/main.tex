%%
%% This is file `sample-sigplan.tex',
%% generated with the docstrip utility.
%%
%% The original source files were:
%%
%% samples.dtx  (with options: `sigplan')
%% 
%% IMPORTANT NOTICE:
%% 
%% For the copyright see the source file.
%% 
%% Any modified versions of this file must be renamed
%% with new filenames distinct from sample-sigplan.tex.
%% 
%% For distribution of the original source see the terms
%% for copying and modification in the file samples.dtx.
%% 
%% This generated file may be distributed as long as the
%% original source files, as listed above, are part of the
%% same distribution. (The sources need not necessarily be
%% in the same archive or directory.)
%%
%% Commands for TeXCount
%TC:macro \cite [option:text,text]
%TC:macro \citep [option:text,text]
%TC:macro \citet [option:text,text]
%TC:envir table 0 1
%TC:envir table* 0 1
%TC:envir tabular [ignore] word
%TC:envir displaymath 0 word
%TC:envir math 0 word
%TC:envir comment 0 0
%%
%%
%% The first command in your LaTeX source must be the \documentclass command.
\documentclass[sigplan,screen,authorversion,nonacm]{acmart}
\settopmatter{printacmref=false}
%% NOTE that a single column version is required for 
%% submission and peer review. This can be done by changing
%% the \doucmentclass[...]{acmart} in this template to 
%% \documentclass[manuscript,screen,review]{acmart}
%% 
%% To ensure 100% compatibility, please check the white list of
%% approved LaTeX packages to be used with the Master Article Template at
%% https://www.acm.org/publications/taps/whitelist-of-latex-packages 
%% before creating your document. The white list page provides 
%% information on how to submit additional LaTeX packages for 
%% review and adoption.
%% Fonts used in the template cannot be substituted; margin 
%% adjustments are not allowed.
%%
%% \BibTeX command to typeset BibTeX logo in the docs
\AtBeginDocument{%
  \providecommand\BibTeX{{%
    \normalfont B\kern-0.5em{\scshape i\kern-0.25em b}\kern-0.8em\TeX}}}

%% Rights management information.  This information is sent to you
%% when you complete the rights form.  These commands have SAMPLE
%% values in them; it is your responsibility as an author to replace
%% the commands and values with those provided to you when you
%% complete the rights form.
\setcopyright{acmcopyright}
\copyrightyear{2018}
\acmYear{2018}
\acmDOI{XXXXXXX.XXXXXXX}

%% These commands are for a PROCEEDINGS abstract or paper.

%
%  Uncomment \acmBooktitle if th title of the proceedings is different
%  from ``Proceedings of ...''!
%
%\acmBooktitle{Woodstock '18: ACM Symposium on Neural Gaze Detection,
%  June 03--05, 2018, Woodstock, NY} 



%%
%% Submission ID.
%% Use this when submitting an article to a sponsored event. You'll
%% receive a unique submission ID from the organizers
%% of the event, and this ID should be used as the parameter to this command.
%%\acmSubmissionID{123-A56-BU3}

%%
%% For managing citations, it is recommended to use bibliography
%% files in BibTeX format.
%%
%% You can then either use BibTeX with the ACM-Reference-Format style,
%% or BibLaTeX with the acmnumeric or acmauthoryear sytles, that include
%% support for advanced citation of software artefact from the
%% biblatex-software package, also separately available on CTAN.
%%
%% Look at the sample-*-biblatex.tex files for templates showcasing
%% the biblatex styles.
%%

%%
%% The majority of ACM publications use numbered citations and
%% references.  The command \citestyle{authoryear} switches to the
%% "author year" style.
%%
%% If you are preparing content for an event
%% sponsored by ACM SIGGRAPH, you must use the "author year" style of
%% citations and references.
%% Uncommenting
%% the next command will enable that style.
%%\citestyle{acmauthoryear}

%%
%% end of the preamble, start of the body of the document source.
\begin{document}

%%
%% The "title" command has an optional parameter,
%% allowing the author to define a "short title" to be used in page headers.
\title{Report: [Title of your paper]}

%%
%% The "author" command and its associated commands are used to define
%% the authors and their affiliations.
%% Of note is the shared affiliation of the first two authors, and the
%% "authornote" and "authornotemark" commands
%% used to denote shared contribution to the research.
\author{Your Name}
\email{studentAsemail@ucr.edu}
\affiliation{%
  \institution{University of California, Riverside}
}





%%
%% By default, the full list of authors will be used in the page
%% headers. Often, this list is too long, and will overlap
%% other information printed in the page headers. This command allows
%% the author to define a more concise list
%% of authors' names for this purpose.

%%
%% The abstract is a short summary of the work to be presented in the
%% article.
\begin{abstract}
Your paper review should offer a comprehensive understanding of the research paper you presented.
You should provide insights into the paper's primary problem, technical details, and evaluation, focusing on its strengths and significant contributions.
We expect you to include the sections as shown in this template.
You also need to critically discuss this paper, too, for example, evaluate its contribution in the context of your entire session.
Make sure you cover the unaddressed questions raised by the audience or me during the presentation.
The report needs to include at least one (1) page.


\end{abstract}

%%
%% The code below is generated by the tool at http://dl.acm.org/ccs.cfm.
%% Please copy and paste the code instead of the example below.
%%




%%
%% This command processes the author and affiliation and title
%% information and builds the first part of the formatted document.
\maketitle




\section{Overview, Background, and Motivation}
In this section, your goal is to provide a comprehensive summary of the paper you presented.
Try answering the following questions in your write-up for this section:
\begin{itemize}
    \item What's the problem that the authors are trying to solve?
    \item Why is this problem important? Would solving this problem have any practical impact?
    \item What are the conventional approaches (if any), and why they don't work?
\end{itemize}



\section{Technical Details}
This section describes the technical details about this paper.
You are expected to provide a high-level overview first, for instance, what's the idea behind the solution, what's the overall procedure of the algorithm, etc.
You could then expand some part of the technical details that you find especially useful or interesting.


Describe how paper implement and evaluate the solution, highlighting the testbed and the metrics used to showcase the effectiveness of the 
solution.




\section{Critical Comments}
\subsection{Contributions to the Field}
You will highlight the work's major contribution and novelty.
Put it into context of the session and showcase how this work advances the entire fields that are relevant.


\subsection{Limitations}
The involved paper/technology has already been published, but it might have certain limitations. 
For example, it only works in certain scenarios while cannot be applied to others.
Identify these limitations and describe them.


\subsection{Future Directions}
In this section, you can focus on emphasizing the potential for future research in the broader context of the paper's subject matter.


\end{document}
