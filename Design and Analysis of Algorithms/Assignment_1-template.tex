\documentclass{article}[12pt]
\usepackage{fullpage}
\usepackage{algorithm}
\usepackage{algpseudocode}
\usepackage{multicol}
\usepackage{amsmath}
\usepackage{amsfonts}
\usepackage{amsthm}
\usepackage{amssymb}
\usepackage{url}
\usepackage{enumitem}
\usepackage{tikz}
\usepackage{graphicx}
\usepackage{framed}


\newcommand{\deadline}{11:59pm, Friday, 01/19}
\newcommand{\programmingdeadline}{11:59pm, Friday, 01/19}

\newcommand{\assigntitle}[1]{{
  \noindent \large \bf
  CS218, Winter 2024, \hfill Due: {\deadline}\\
  Assignment \##1 \footnote{Some of the problems are adapted from existing problems from online sources. Thanks to the original authors.}\\
  Name: %put in your name here
  \hspace{3.2in}
  Student ID: %put in your id here
  \\
  [-.05in]
  \mbox{}\hrulefill \mbox{}\\}}

\begin{document}

\assigntitle{1}{}
\date{}



\section{A Complex Complexity Problem (1.2pts)}
\begin{center}
\begin{enumerate}
  \item $(\sqrt{3})^{\log n}=$ \underline{~~~~}$(n)$\\ Explain briefly:
  %\item $3^n=$ \underline{~~~~}$(n!)$\\ Explain briefly:
  \item $\log\log n=$ \underline{~~~~}$(\sqrt{\log n})$\\ Explain briefly:
  %\item $\log(n!)=$ \underline{~~~~}$(n/\log n)$\\Explain briefly:
  \item $\log (n!)=$ \underline{~~~~}$(n\log n)$\\ Explain briefly:
  \item $2^n=$\underline{~~~~}$(3^n)$\\ Explain briefly: \\ ~\\
  %\item $\sqrt{\log n}=$\underline{~~~~}$(\log \sqrt{n})$\\ Explain briefly:
\end{enumerate}
\end{center}


\section{Solve Recurrences (0.6pts)}
For all of them, you can assume the base case is when $n$ is a constant, $T(n)$ is also a constant. Use $\Theta(\cdot)$ to present your answer.

\begin{enumerate}
  %\item $T(n)=T(\sqrt{n})+1$
  \item $T(n)=T(n/2)+n\log n$
  \item $T(n)=2T(n/4)+\sqrt{n}$
  \item $T(n)=4T(n/4) + {n}^{1/2}$
\end{enumerate}

\section{Test the candies (2.2 pts + 1 candy)}

\begin{enumerate}
  \item (0.2pts) 
  \item (0.2pts) 
  \item (0.6pts) 
  \item (0.2pts) 
  \item (0.3pts) 
  \item (0.7pts) 
  \item (bonus, 0.5pts and 1 candy) 
\end{enumerate}
\newpage

\newpage

\section{(Training) Finding the Minimum Value (1.5 pts)}


\section{(Programming) Sort the Train (1 + 1 pts)}


\subsection{Programming (1 pt) and Hints}

\subsection{(Bonus) Sort the Train - Let's prove it! (1 pt bonus + 2 candies)}

\begin{enumerate}
  \item (0.3pts + 1 candy) 
  \item (0.3pts) 
  \item (0.4pts + 1 candy) 
\end{enumerate}


\section{Being Unique (1 pt bonus + 2 candies)}



\end{document}
