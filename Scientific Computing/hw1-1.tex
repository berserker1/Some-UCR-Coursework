\documentclass{article}
\usepackage[cm]{fullpage}
\usepackage{hyperref}
\usepackage{amsmath}
\usepackage{amssymb}
\usepackage{amsthm}
\usepackage{dsfont}
\usepackage{mathrsfs}
\usepackage{bm}
\usepackage{enumitem}

\renewcommand{\vec}[1]{\mathbf{#1}}
\newcommand{\gp}[1]{\left( #1 \right)}
\newcommand{\mx}[1]{\begin{bmatrix} #1 \end{bmatrix}}
\newcommand{\abs}[1]{\left| #1 \right|}
\newcommand{\DIV}[1]{\nabla \cdot {#1}}
\newcommand{\V}[1]{\mathbf{#1}}
\newcommand{\mthree}[9]{
  \begin{pmatrix}
    #1 & #2 & #3 \\
    #4 & #5 & #6 \\
    #7 & #8 & #9 \\
  \end{pmatrix}
  }

\DeclareMathOperator{\tr}{tr}

\title{Homework 1}
\date{}

\begin{document}
\maketitle{}


\subsection*{Review of matrix and vector algebra}

Some definitions for the following problems.  Let $\alpha, \beta, \gamma, \delta \in \mathbb{R}$ be scalars, $ \vec{u}, \vec{v}, \vec{w},\vec{a}, \vec{b} \in \mathbb{R}^{n}$ be vectors, and  $A, B \in \mathbb{R}^{n \times n}, C \in \mathbb{R}^{m \times n}$ be matrices.  Let the $j^{th}$ \emph{column} of $A$ be denoted by $\vec{a}_j$, so that $A  = \mx { \vec{a}_1 & \vec{a}_2 & \ldots & \vec{a}_n} $.
Similarly, let the $i^{th}$ \emph{row} of $B$ be denoted by $\vec{b}_i^T$, so that $B = \mx { \vec{b}_1^T \\ \vdots \\ \vec{b}_n^T}$.  Let $a_{ij}$ denote the entry of $A$ in the $i^{th}$ row and $j^{th}$ column.

Let $\vec{e}_i$ represent the $i^{th}$ canonical (standard basis) vector, every entry of which is 0, except for the $i^{th}$ entry which is 1.  E.g., $\vec{e}_1 = (1, 0, \ldots, 0)^T$, $\vec{e}_2 = (0, 1, 0, \ldots, 0)^T$.

\begin{enumerate}

\item It is important to stay aware of the dimensions of things, both as a sanity check and for understanding.  These problems give you practice with this.  

For each problem, state the size of the resultant quantity (e.g., $1 \times 1$, $1 \times 3$, ...) .  You do \underline{not} need to compute the result.
  \begin{enumerate}
    \item $ \mx{1 & 2 & 3} \mx{4 \\ 5 \\ 6 } $
    \item $  \mx{4 \\ 5 \\ 6 }\mx{1 & 2 & 3} $
    \item $  \mx{1 & 2 & 3 \\ 1 & 2 & 2} \mx{4 \\ 5 \\ 6 } $
    \item $  \mx{1 & 2 & 3 \\ 1 & 2 & 2} \mx{1 & 4 \\ 2& 5 \\ 3 & 6 } $
    \item $  \mx{1 & 2 & 3}  \mx{1 & 4 & -1 \\ 2& 5 & 0 \\ 3 & 6 &1 } \mx{4 \\ 5 \\ 6} $
    \item $  \mx{1 & 2 & 3}  \mx{1 & 4 \\ 2& 5 \\ 3 & 6 } \mx{4 \\ 5} $
  \end{enumerate}
  \begin{enumerate}
    \setcounter{enumii}{6}
    \item $\alpha \vec{v}$
    \item $\alpha B$
    \item $\vec{v}^T \vec{w}$
    \item $A \vec{v}$
    \item $C \vec{v}$
    \item $\vec{v}^T A \vec{w}$
    \item $\vec{v}^T C \vec{v}$
    \item $\vec{v}^T C^T C \vec{v}$
  \end{enumerate}
\item Compute these \underline{inner} products:
  \begin{enumerate}
    \item $\mx{1 & 2 & 3} \mx{1 \\ 0 \\ 0} = $
    \item $\mx{1 & 2 & 3} \mx{-1 \\ 0 \\ 0} = $
    \item $\mx{1 & 2 & 3} \mx{1 \\ 1 \\ 1} = $
  \end{enumerate}
\item Compute these \underline{outer} products:
  \begin{enumerate}
    \item $\mx{1 \\ 0 \\ 0} \mx{1 & 2 & 3}  = $
    \item $\mx{-1 \\ 0 \\ 0} \mx{1 & 2 & 3}  = $
    \item $\mx{1 \\ 1 \\ 1} \mx{1 & 2 & 3}  = $
    \item $\mx{4 \\ -1 \\ 10} \mx{1 & 2 & 3}  = $
  \end{enumerate}
\item It is useful to know the result of a matrix times a canonical vector.
\begin{enumerate}
\item $\mx{ 1 & 2 & 3\\ 4& 5 & 6} \mx{1 \\ 0 \\ 0} = $
\item $\mx{ 1 & 2 & 3\\ 4& 5 & 6} \mx{0 \\ 0 \\ 1} = $
\end{enumerate}
\begin{enumerate}
  \setcounter{enumii}{2}
  \item $ A \vec{e}_1 =$
  \item $ A \vec{e}_2 =$
  \item $ A \vec{e}_n =$
\end{enumerate}
\item Similar results hold for a canonical row vector times a matrix. Find the following.
  \begin{enumerate}
  \item $\mx{1 & 0} \mx{ 1 & 2 & 3\\ 4& 5 & 6}  = $
  \item $\mx{0 & 1} \mx{ 1 & 2 & 3\\ 4& 5 & 6}  = $
  \item $\vec{e}_1^T B = $
  \item $\vec{e}_j^TB  = $
  \end{enumerate}
\item Is it true in general that $\vec{v} \vec{u}^T A \vec{w}  =  (\vec{u}^T A \vec{w})\vec{v}$? Explain.
\item Is it true in general that  $AB = BA$?  Explain if true, or give a counterexample if false.
\item What is the value of  $\vec{e}_i^T A \vec{e}_j$ in terms of the entries of $A$?
\item It is very helpful for matrix algorithms to be able to think of matrix-matrix multiplication $C = AB$ in two different ways.  The first is the way you were probably taught: The entries of the result $C$ are $c_{ij} = \vec{a}_i^T \vec{b}_j$, where $\vec{a}_i^T$ are rows of $A$ and $\vec{b}_j$ are columns of $B$.  The second is as a sum of outer products $C = \sum_{j=i=1}^n \vec{a}_j \vec{b}_i^T$, where $\vec{a}_j$ are columns of $A$ and $\vec{b}_i^T$ are rows of $B$.  To practice these, do the following.
  \begin{enumerate}
  \item \emph{Inner products}. Compute using inner products,
    $$
  \mx{ 1 & 2 \\ 3 & 4 \\ -1 & 0} \mx{1 & -1 &1 \\ 0 & 1 & 2} = 
  $$
  \item \emph{Outer products}. Verify that you get the same result using outer products (show the two intermediate matrices),
$$
    \mx{ 1  \\ 3 \\ -1} \mx{1 & -1 &1 } +   \mx{  2 \\  4 \\ 0} \mx{0 & 1 & 2 } = 
  $$
\end{enumerate}

%\setcounter{enumi}{9}
\item Show that $(AB)^T=B^T A^T$.  
\item Assume $A$ and $B$ are invertible.  Show that $(A B)^{-1} = B^{-1} A^{-1} $.
\item Let $C \in \mathbb{R}^{m \times n}$, where $m < n$, and $C$ is full rank so that $CC^T$ is invertible.  Let $P = I - C^T(CC^T)^{-1}C$.    
  \begin{enumerate}
  \item Simplify $CP$.
  \item Show that $PP=P$.
  \end{enumerate}
\item Let $A$ be symmetric, i.e., $A = A^T$.  Which of the following are necessarily also symmetric? Why?
  \begin{enumerate}
  \item $A^{-1}$
  \item  $A^T$
  \item $A^2$
  \end{enumerate}
  


\end{enumerate}

\end{document}
